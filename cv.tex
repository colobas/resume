%!TEX TS-program = xelatex
\documentclass[]{colobas-cv}
\usepackage{scrextend}

\begin{document}
\header{Guilherme}{Pires}{ML Engineer}

\begin{minipage}[t]{0.3\textwidth}
\section{find me}
  \vspace{7pt}
  \href{mailto:mail@gpir.es}{mail@gpir.es}\\
  \href{https://colobas.github.io}{colobas.github.io}\\
  \href{https://github.com/colobas}{github.com/colobas}\\
  \href{https://linkedin.com/in/guilhermegrijopires/}{linkedin.com/in/guilhermegrijopires}\\
  PT: +351 91 504 94 34\\
  %US: +1 662 617 0047%
\end{minipage}
\hfill
\begin{minipage}[t]{0.3\textwidth}
\section{skills \& tools}
  \vspace{7pt}
  \textbf{very confident:} python, pandas, sklearn, numpy, bash, linux\\
  \textbf{confident:} pytorch, kubernetes, azure, aws, golang, c, dask, docker\\
  \textbf{beginner:} scala, nlp techniques, mesos, rabbitmq, gcp%
\end{minipage}
\hfill
\begin{minipage}[t]{0.3\textwidth}
\section{technical interests}
  \vspace{7pt}
  ml theory, probabilistic modelling, variational inference, optimization%
  \vspace{15pt}
\section{personal interests}
  \vspace{7pt}
  technology, hackathons, climate change, economy, philosophy, the meaning of life,
  music, running
\end{minipage}

\vspace{0.5cm}

\section{education}
%\begin{entrylist}
\entry
  {Since Sep 2012}
  {MSc degree}
  {IST - Lisboa}
  {Electrical and Computer Engineering}
\entry
  {Oct - Feb 2015/16}
  {ERASMUS - exchange program}
  {KIT - Karlsruhe}
  {Computer Science}
%\end{entrylist}

\section{experience}
%\begin{entrylist}
\entry
  {Since Sep 2017}
  {\href{https://jungle.ai}{Jungle}}
  {ML Engineer}
  {Since Sep-2017 I'm a full time \textbf{ML Engineer}. I've worked with diverse 
  \textbf{multivariate time-series} datasets, with data sources spanning from 
  \textbf{heavy industry}, to \textbf{utilities} and \textbf{renewables}.
  Keywords: machine learning, time-series, root cause analysis, forecasting,
  probabilistic modelling}
\entry
  {Jul - Sep 2017}
  {\href{https://jungle.ai}{Jungle}}
  {Data Engineering Intern}
  {I started at Jungle as a \textbf{Data Engineering} intern. During my internship I
  worked on assembling an \textbf{infrastructure} to enable Machine Learning at scale.
  To do so, I looked into and gained experiece with technologies such as
  \textbf{Mesos, Marathon, Kubernetes, Dask, Docker}. I also built a PoC of a
  job scheduler, see \href{https://github.com/colobas/obras}{here}}
\entry
  {Jan - Mar 2017}
  {\href{https://snowplowanalytics.com}{Snowplow}}
  {Data Engineering Intern}
  {I was a remote "wintern" at Snowplow for about 3 months in 2017. I helped
  them extend some of their tech to \textbf{Google Cloud Platform}, having gained 
  introductory experience with \textbf{Scala}, and GCP along the way. See the
  resulting blogpost \href{https://snowplowanalytics.com/blog/2017/03/30/google-cloud-dataflow-example-project-released/}{here}}
\entry
  {Aug - Oct 2013}
  {\href{https://quidgest.com}{Quidgest}}
  {SW Engineering Intern}
  {My first job in software was a summer internship at Quidgest. I spent my time
  there review code and refactoring an internal tool, at the R\&D department.
  I worked with C, C\#, C++, SQL}
%\end{entrylist}
\section{projects}

%\begin{entrylist}
\entry
  {}
  {MSc thesis: Deep Generative Models for Multivariate Time-series}
  {Python}
  {I'm working with unsupervised learning for multivariate time-series. I'm
  specially interested in segmenting high dimensional time-series into
  "regimes of operation", similar to the goal of changepoint detection.}
\entry
  {}
  {PoC of a job scheduler}
  {Python}
  {During my summer internship at Jungle, I built a PoC of a job scheduler,
  using RabbitMQ for task queues, and Redis for logging. The goal was to have
  the concept of Recipes - which would be descriptions of tasks. Basically,
  to create a Recipe you'd implement a subclass of the AbstractRecipe class.
  The worker spawners are also abstract, to allow workers of different natures.
  I only implemented a CPUSpawner which creates workers in different processes.}
\entry
  {}
  {Scraper for politifact.com claim checks}
  {Python}
  {I was working on an automated claim checking project, and needed a dataset,
  so I built a scraper for \href{politifact.com}{politifact.com}}
\entry
  {}
  {Betfair scraper and automated betting bot}
  {Python}
  {Some friends came up with a soccer betting strategy, based on metrics 
  calculated from live statistics of soccer matches. I scraped Betfair and some
  other websites, and built a bot that computed the metrics and placed bets
  according to the strategy.}
\entry
  {}
  {Dynamic Bayesian Network learning}
  {Java}
  {Implemented a program to learn DBNs from data. See project assigment 
  \href{https://fenix.tecnico.ulisboa.pt/downloadFile/1689468335554723/apresentacao-projecto-POO1415.pdf}{here}}
\entry
  {}
  {SATPLAN system}
  {Python}
  {Implemented a SATPLAN system for an AI course. It worked by translating a
  planning problem into a SAT problem, solving the SAT problem, and translating
  the solution into the planning semantics. See here \href{https://github.com/colobas/ia-proj2}{here}}
\entry
  {}
  {PoC of a distributed "e-mail" system}
  {Go}
  {Implemented a distributed system for nodes to send private messages. It's a
  combination of a distributed hashtable, to store messages in the network, and
  an onion routing circuit, so that the node who stores the message is not the
  node who wants to send it. See \href{https://github.com/colobas/distributed-email}{here}}
\entry
  {}
  {PoC of a distributed hashtable}
  {C}
  {Implemented a skeleton for a distributed hashtable.}
\entry
  {}
  {MAXSAT solver - serial, parallel, distributed}
  {C}
  {Implemented three versions of a MAXSAT solver: serial, parallel (OpenMP), and
  distributed (MPI). See \href{https://github.com/colobas/maxsat-solver}{here}}
%\end{entrylist}
\end{document}

